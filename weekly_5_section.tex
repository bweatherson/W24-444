% Options for packages loaded elsewhere
\PassOptionsToPackage{unicode}{hyperref}
\PassOptionsToPackage{hyphens}{url}
\PassOptionsToPackage{dvipsnames,svgnames,x11names}{xcolor}
%
\documentclass[
  12pt,
  letterpaper,
  DIV=11,
  numbers=noendperiod]{scrartcl}

\usepackage{amsmath,amssymb}
\usepackage{iftex}
\ifPDFTeX
  \usepackage[T1]{fontenc}
  \usepackage[utf8]{inputenc}
  \usepackage{textcomp} % provide euro and other symbols
\else % if luatex or xetex
  \usepackage{unicode-math}
  \defaultfontfeatures{Scale=MatchLowercase}
  \defaultfontfeatures[\rmfamily]{Ligatures=TeX,Scale=1}
\fi
\usepackage{lmodern}
\ifPDFTeX\else  
    % xetex/luatex font selection
  \setmainfont[Scale = MatchLowercase]{Scala Pro}
  \setsansfont[]{Scala Sans Pro}
\fi
% Use upquote if available, for straight quotes in verbatim environments
\IfFileExists{upquote.sty}{\usepackage{upquote}}{}
\IfFileExists{microtype.sty}{% use microtype if available
  \usepackage[]{microtype}
  \UseMicrotypeSet[protrusion]{basicmath} % disable protrusion for tt fonts
}{}
\makeatletter
\@ifundefined{KOMAClassName}{% if non-KOMA class
  \IfFileExists{parskip.sty}{%
    \usepackage{parskip}
  }{% else
    \setlength{\parindent}{0pt}
    \setlength{\parskip}{6pt plus 2pt minus 1pt}}
}{% if KOMA class
  \KOMAoptions{parskip=half}}
\makeatother
\usepackage{xcolor}
\usepackage[top=25mm,left=25mm,heightrounded]{geometry}
\setlength{\emergencystretch}{3em} % prevent overfull lines
\setcounter{secnumdepth}{-\maxdimen} % remove section numbering
% Make \paragraph and \subparagraph free-standing
\ifx\paragraph\undefined\else
  \let\oldparagraph\paragraph
  \renewcommand{\paragraph}[1]{\oldparagraph{#1}\mbox{}}
\fi
\ifx\subparagraph\undefined\else
  \let\oldsubparagraph\subparagraph
  \renewcommand{\subparagraph}[1]{\oldsubparagraph{#1}\mbox{}}
\fi


\providecommand{\tightlist}{%
  \setlength{\itemsep}{0pt}\setlength{\parskip}{0pt}}\usepackage{longtable,booktabs,array}
\usepackage{calc} % for calculating minipage widths
% Correct order of tables after \paragraph or \subparagraph
\usepackage{etoolbox}
\makeatletter
\patchcmd\longtable{\par}{\if@noskipsec\mbox{}\fi\par}{}{}
\makeatother
% Allow footnotes in longtable head/foot
\IfFileExists{footnotehyper.sty}{\usepackage{footnotehyper}}{\usepackage{footnote}}
\makesavenoteenv{longtable}
\usepackage{graphicx}
\makeatletter
\def\maxwidth{\ifdim\Gin@nat@width>\linewidth\linewidth\else\Gin@nat@width\fi}
\def\maxheight{\ifdim\Gin@nat@height>\textheight\textheight\else\Gin@nat@height\fi}
\makeatother
% Scale images if necessary, so that they will not overflow the page
% margins by default, and it is still possible to overwrite the defaults
% using explicit options in \includegraphics[width, height, ...]{}
\setkeys{Gin}{width=\maxwidth,height=\maxheight,keepaspectratio}
% Set default figure placement to htbp
\makeatletter
\def\fps@figure{htbp}
\makeatother

\KOMAoption{captions}{tableheading}
\makeatletter
\@ifpackageloaded{caption}{}{\usepackage{caption}}
\AtBeginDocument{%
\ifdefined\contentsname
  \renewcommand*\contentsname{Table of contents}
\else
  \newcommand\contentsname{Table of contents}
\fi
\ifdefined\listfigurename
  \renewcommand*\listfigurename{List of Figures}
\else
  \newcommand\listfigurename{List of Figures}
\fi
\ifdefined\listtablename
  \renewcommand*\listtablename{List of Tables}
\else
  \newcommand\listtablename{List of Tables}
\fi
\ifdefined\figurename
  \renewcommand*\figurename{Figure}
\else
  \newcommand\figurename{Figure}
\fi
\ifdefined\tablename
  \renewcommand*\tablename{Table}
\else
  \newcommand\tablename{Table}
\fi
}
\@ifpackageloaded{float}{}{\usepackage{float}}
\floatstyle{ruled}
\@ifundefined{c@chapter}{\newfloat{codelisting}{h}{lop}}{\newfloat{codelisting}{h}{lop}[chapter]}
\floatname{codelisting}{Listing}
\newcommand*\listoflistings{\listof{codelisting}{List of Listings}}
\makeatother
\makeatletter
\makeatother
\makeatletter
\@ifpackageloaded{caption}{}{\usepackage{caption}}
\@ifpackageloaded{subcaption}{}{\usepackage{subcaption}}
\makeatother
\ifLuaTeX
  \usepackage{selnolig}  % disable illegal ligatures
\fi
\IfFileExists{bookmark.sty}{\usepackage{bookmark}}{\usepackage{hyperref}}
\IfFileExists{xurl.sty}{\usepackage{xurl}}{} % add URL line breaks if available
\urlstyle{same} % disable monospaced font for URLs
\hypersetup{
  pdftitle={Weekly 5 Section Version},
  pdfauthor={Phil 444},
  colorlinks=true,
  linkcolor={black},
  filecolor={Maroon},
  citecolor={Blue},
  urlcolor={Blue},
  pdfcreator={LaTeX via pandoc}}

\title{Weekly 5 Section Version}
\author{Phil 444}
\date{2024-04-11}

\begin{document}
\maketitle

The questions all concern the signaling game shown below. The game is
like the ones we discussed in class. First Nature reveals a type (A or
B), then Proposer sends a signal (Left or Right), then Responder, seeing
the signal but not the state, chooses and action (Up or Down). The
payout to each player is a function of all three choices, as shown in
both the table and the tree.

\begin{longtable}[]{@{}llll@{}}

\caption{\label{tbl-tree-payouts}Payouts for Weekly 5 section version}

\tabularnewline

\toprule\noalign{}
Type & Proposer & Responder & Payouts \\
\midrule\noalign{}
\endhead
\bottomrule\noalign{}
\endlastfoot
A & L & D & 4, 3 \\
A & L & U & 0, 3 \\
A & R & D & 3, 1 \\
A & R & U & 2, 2 \\
B & L & D & 3, 1 \\
B & L & U & 4, 3 \\
B & R & D & 4, 3 \\
B & R & U & 1, 0 \\

\end{longtable}

\newpage

In this tree, Proposer has four possible strategies:

\begin{enumerate}
\def\labelenumi{\arabic{enumi}.}
\tightlist
\item
  Left if A, Left if B (LL)
\item
  Left if A, Right if B (LR)
\item
  Right if A, Left if B (RL)
\item
  Right if A, Right if B (RR)
\end{enumerate}

And Responder has four possible strategies

\begin{enumerate}
\def\labelenumi{\arabic{enumi}.}
\tightlist
\item
  Up if Left, Up if Right (UU)
\item
  Up if Left, Down if Right (UD)
\item
  Down if Left, Up if Right (DU)
\item
  Down if Left, Down if Right (DD)
\end{enumerate}

That leads to 16 possible combinations of strategies. For each of these
16, work out

\begin{enumerate}
\def\labelenumi{\Alph{enumi}.}
\tightlist
\item
  What Proposer's \emph{expected} payout is.
\item
  What Responder's \emph{expected} payout is.
\end{enumerate}

Once you've done that, for each pair work out whether it is:

\begin{enumerate}
\def\labelenumi{\Alph{enumi}.}
\tightlist
\item
  A pooling equilibrium;
\item
  A separating equilibrium; or
\item
  Not an equilibrium.
\end{enumerate}

\subsection{Answers}\label{answers}

\begin{longtable}[]{@{}lllll@{}}

\caption{\label{tbl-main-tree}Expected values for Weekly 1 section
version}

\tabularnewline

\toprule\noalign{}
P1 & DD & DU & UD & UU \\
\midrule\noalign{}
\endhead
\bottomrule\noalign{}
\endlastfoot
LL & 3.6, 2.2 & 3.6, 2.2 & 1.6, 3 & 1.6, 3 \\
LR & 4, 3 & 2.8, 1.8 & 1.6, 3 & 0.4, 1.8 \\
RL & 3, 1 & 2.4, 1.6 & 3.4, 1.8 & 2.8, 2.4 \\
RR & 3.4, 1.8 & 1.6, 1.2 & 3.4, 1.8 & 1.6, 1.2 \\

\end{longtable}

The only pooling equilibrium is:

\begin{itemize}
\tightlist
\item
  RR, UD
\end{itemize}

The separating equilibria are:

\begin{itemize}
\tightlist
\item
  LR, DD
\item
  RL, UU
\end{itemize}

Note that in general there may be 0, 1, or more of each type of
equilibrium.



\end{document}
