% Options for packages loaded elsewhere
\PassOptionsToPackage{unicode}{hyperref}
\PassOptionsToPackage{hyphens}{url}
\PassOptionsToPackage{dvipsnames,svgnames,x11names}{xcolor}
%
\documentclass[
  12pt,
  letterpaper,
  DIV=11,
  numbers=noendperiod]{scrartcl}

\usepackage{amsmath,amssymb}
\usepackage{iftex}
\ifPDFTeX
  \usepackage[T1]{fontenc}
  \usepackage[utf8]{inputenc}
  \usepackage{textcomp} % provide euro and other symbols
\else % if luatex or xetex
  \usepackage{unicode-math}
  \defaultfontfeatures{Scale=MatchLowercase}
  \defaultfontfeatures[\rmfamily]{Ligatures=TeX,Scale=1}
\fi
\usepackage{lmodern}
\ifPDFTeX\else  
    % xetex/luatex font selection
  \setmainfont[Scale = MatchLowercase]{Scala Pro}
  \setsansfont[]{Scala Sans Pro}
\fi
% Use upquote if available, for straight quotes in verbatim environments
\IfFileExists{upquote.sty}{\usepackage{upquote}}{}
\IfFileExists{microtype.sty}{% use microtype if available
  \usepackage[]{microtype}
  \UseMicrotypeSet[protrusion]{basicmath} % disable protrusion for tt fonts
}{}
\makeatletter
\@ifundefined{KOMAClassName}{% if non-KOMA class
  \IfFileExists{parskip.sty}{%
    \usepackage{parskip}
  }{% else
    \setlength{\parindent}{0pt}
    \setlength{\parskip}{6pt plus 2pt minus 1pt}}
}{% if KOMA class
  \KOMAoptions{parskip=half}}
\makeatother
\usepackage{xcolor}
\setlength{\emergencystretch}{3em} % prevent overfull lines
\setcounter{secnumdepth}{-\maxdimen} % remove section numbering
% Make \paragraph and \subparagraph free-standing
\ifx\paragraph\undefined\else
  \let\oldparagraph\paragraph
  \renewcommand{\paragraph}[1]{\oldparagraph{#1}\mbox{}}
\fi
\ifx\subparagraph\undefined\else
  \let\oldsubparagraph\subparagraph
  \renewcommand{\subparagraph}[1]{\oldsubparagraph{#1}\mbox{}}
\fi


\providecommand{\tightlist}{%
  \setlength{\itemsep}{0pt}\setlength{\parskip}{0pt}}\usepackage{longtable,booktabs,array}
\usepackage{calc} % for calculating minipage widths
% Correct order of tables after \paragraph or \subparagraph
\usepackage{etoolbox}
\makeatletter
\patchcmd\longtable{\par}{\if@noskipsec\mbox{}\fi\par}{}{}
\makeatother
% Allow footnotes in longtable head/foot
\IfFileExists{footnotehyper.sty}{\usepackage{footnotehyper}}{\usepackage{footnote}}
\makesavenoteenv{longtable}
\usepackage{graphicx}
\makeatletter
\def\maxwidth{\ifdim\Gin@nat@width>\linewidth\linewidth\else\Gin@nat@width\fi}
\def\maxheight{\ifdim\Gin@nat@height>\textheight\textheight\else\Gin@nat@height\fi}
\makeatother
% Scale images if necessary, so that they will not overflow the page
% margins by default, and it is still possible to overwrite the defaults
% using explicit options in \includegraphics[width, height, ...]{}
\setkeys{Gin}{width=\maxwidth,height=\maxheight,keepaspectratio}
% Set default figure placement to htbp
\makeatletter
\def\fps@figure{htbp}
\makeatother

\KOMAoption{captions}{tableheading}
\makeatletter
\@ifpackageloaded{caption}{}{\usepackage{caption}}
\AtBeginDocument{%
\ifdefined\contentsname
  \renewcommand*\contentsname{Table of contents}
\else
  \newcommand\contentsname{Table of contents}
\fi
\ifdefined\listfigurename
  \renewcommand*\listfigurename{List of Figures}
\else
  \newcommand\listfigurename{List of Figures}
\fi
\ifdefined\listtablename
  \renewcommand*\listtablename{List of Tables}
\else
  \newcommand\listtablename{List of Tables}
\fi
\ifdefined\figurename
  \renewcommand*\figurename{Figure}
\else
  \newcommand\figurename{Figure}
\fi
\ifdefined\tablename
  \renewcommand*\tablename{Table}
\else
  \newcommand\tablename{Table}
\fi
}
\@ifpackageloaded{float}{}{\usepackage{float}}
\floatstyle{ruled}
\@ifundefined{c@chapter}{\newfloat{codelisting}{h}{lop}}{\newfloat{codelisting}{h}{lop}[chapter]}
\floatname{codelisting}{Listing}
\newcommand*\listoflistings{\listof{codelisting}{List of Listings}}
\makeatother
\makeatletter
\makeatother
\makeatletter
\@ifpackageloaded{caption}{}{\usepackage{caption}}
\@ifpackageloaded{subcaption}{}{\usepackage{subcaption}}
\makeatother
\ifLuaTeX
  \usepackage{selnolig}  % disable illegal ligatures
\fi
\IfFileExists{bookmark.sty}{\usepackage{bookmark}}{\usepackage{hyperref}}
\IfFileExists{xurl.sty}{\usepackage{xurl}}{} % add URL line breaks if available
\urlstyle{same} % disable monospaced font for URLs
\hypersetup{
  pdftitle={PHIL 444: Groups and Choices},
  pdfauthor={Brian Weatherson},
  colorlinks=true,
  linkcolor={black},
  filecolor={Maroon},
  citecolor={Blue},
  urlcolor={Blue},
  pdfcreator={LaTeX via pandoc}}

\title{PHIL 444: Groups and Choices}
\usepackage{etoolbox}
\makeatletter
\providecommand{\subtitle}[1]{% add subtitle to \maketitle
  \apptocmd{\@title}{\par {\large #1 \par}}{}{}
}
\makeatother
\subtitle{Winter 2024}
\author{Brian Weatherson}
\date{}

\begin{document}
\maketitle

\textbf{Lead Instructor}: Brian Weatherson\\
\includegraphics[width=1em,height=1em]{syllabus-phil444-w24_files/figure-pdf/fa-icon-a7ff419e70980f9f1a65816048d94526.pdf}
weath@umich.edu\\
\includegraphics[width=1.25em,height=1em]{syllabus-phil444-w24_files/figure-pdf/fa-icon-69b5e588a8cebb8cb21405167a7066e5.pdf}
canvas.umich.edu\\
\strut \\
\textbf{Office Hours}: TBC\\
\strut \\
\textbf{Discussion Section Leades}:\\
- Walla Mohamedali\\
- Brandon Swinney

\section{Course Description}\label{course-description}

This course has four units.

\begin{description}
\tightlist
\item[Group Attitudes]
We'll look at how groups act, whether groups have beliefs, and what it
takes for those beliefs to be reasonable, and how to combine individual
attitudes into a group attitude.
\item[Voting]
We'll look at voting systems that are used around the world, and some
theoretical results about the limits of voting systems.
\item[Games and Coordination]
We'll introduce the basics of game theory, with a focus on (a)
coordination games, and (b) how well empirical evidence matches up to
the theoretical predictions that game theory makes.
\item[Information Networks]
We'll look at some famous games involving transmission or suppression of
information, and link this to contemporary work on optimal information
networks.
\end{description}

\section{Canvas}\label{canvas}

There is a Canvas site for this course, which can be accessed from
\url{https://canvas.umich.edu}. Course documents (syllabus, lecture
notes, assignments) will be available from this site. Please make sure
that you can access this site. Consult the site regularly for
announcements, including changes to the course schedule. And there are
many tools on the site to communicate with each other, and with me.

\section{Required Materials}\label{required-materials}

There is a textbook for the course, which is available for free online.

\begin{itemize}
\tightlist
\item
  \emph{Game Theory} by Giacomo Bonanno, available at
  \url{http://faculty.econ.ucdavis.edu/faculty/bonanno/GT_Book.html}.
\end{itemize}

The other readings will all be available through the university library,
and will be linked on Canvas.

\section{Course Requirements}\label{course-requirements}

\begin{enumerate}
\def\labelenumi{\arabic{enumi}.}
\tightlist
\item
  Do the readings! Nothing I say in class will be more important than
  the reading.
\item
  Come to lectures, and engage. This class will use \textbf{iClicker},
  and you have to get a working iClicker early in term. Later in term,
  we will be doing several games in class to test how well theoretical
  predictions match up with behavior. Even though it's a big lecture
  hall, you are \textbf{encouraged} to ask questions during lecture. I
  do not want to be lecturing for 80 minutes.
\item
  Participate in the discussion sections. It is really important that
  you interact with the discussion section leader. Some of this material
  is hard, and you can only grasp it by talking it through in small
  groups. Some of the material doesn't look hard, but like learning a
  foreign language, you only figure out what you're missing when you try
  to put it into practice.
\item
  Write short papers (about 4 pages, or about 1200 words) on each of the
  first two units of the course.
\item
  Complete 5 of the 6 weekly assignments from parts 3 and 4 of the
  course. (You should complete all six, but only the best five will
  count for credit - this is to allow you to drop one if there is some
  emergency one week. If the future resembles the recent past.)
\item
  Do the final exam, which will be fairly short, and held in the exam
  period.
\end{enumerate}

Both of the papers may be co-written with one other student in the
class. But you may not co-author both papers with the same other
student.

\section{Summary of Grading System}\label{summary-of-grading-system}

\begin{enumerate}
\def\labelenumi{\arabic{enumi}.}
\tightlist
\item
  Two papers - 20\% each, 40\% total.
\item
  Weekly assignments - 6\% each, 5 assignments (that count), 30\% total
\item
  Participation in lecture (including performance in games) - 10\%
\item
  Final Exam - 20\%
\end{enumerate}

\section{Plagiarism}\label{plagiarism}

Although team-work, and even co-authorship, is encouraged, plagiarism is
strictly prohibited. You are responsible for making sure that none of
your work is plagiarized. Be sure to cite work that you use, both direct
quotations and paraphrased ideas. Any citation method that is tolerably
clear is permitted, but if you'd like a good description of a citation
scheme that works well in philosophy, look at
\url{https://www.mendeley.com/guides/apa-citation-guide/}.

You are encouraged to discuss the course material, including
assignments, with your classmates, but all written work that you hand in
under your own name must be your own. If work is handed is as the work
of two people, you are affirming that each person did a fair share of
the work. (Note that when you're submitting work on Canvas, you have to
each submit the paper, even if it is co-authored. That way Canvas knows
that everyone has turned in work.)

You should also be familiar with the academic integrity policies of the
College of Literature, Science \& the Arts at the University of
Michigan, which are available here:
\url{https://lsa.umich.edu/lsa/academics/academic-integrity.html}.
Violations of these policies will be reported to the Office of the
Assistant Dean for Student Academic Affairs, and sanctioned with a
course grade of F.

\section{Collaboration}\label{collaboration}

As noted above, you are allowed to collaborate with other people in the
class. It's a class on group action; we want to encourage group
activity! And in any case, I don't think outside of academia writing
long things by yourself is that important a skill. But there are still a
few constraints on this.

\begin{enumerate}
\def\labelenumi{\arabic{enumi}.}
\tightlist
\item
  You must describe (ideally in footnotes to the document) what ways you
  collaborated with others.
\item
  If you submit co-authored essays, then you should submit a separate
  document setting out what contribution each person made. This doesn't
  have to be long; a paragraph is fine. And there are a lot of ways to
  divide up the work that are reasonable. In particular, you can divide
  things up by parts (e.g., I wrote the first two pages, my collaborator
  wrote the other two), or by temporal stages (e.g., I wrote the first
  draft, my collaborator wrote the final draft). But we want you to say,
  on your own, what the division was.
\item
  You can only co-write papers with other people who have the same GSI;
  otherwise it complicates the grading too much.
\item
  You cannot have the same co-author for each of the two papers.
\end{enumerate}

\section{Citations Policy}\label{citations-policy}

We don't have a standard format for citations; any format will do. But
you do need to cite things. In particular, whenever you say that an
author says something, we want you to cite \textbf{the page} they say it
on. This can either be in a footnote, or just writing the page number in
brackets in the text. This is for three reasons.

\begin{enumerate}
\def\labelenumi{\arabic{enumi}.}
\tightlist
\item
  It's good practice to record this kind of detail. Academic papers are
  long, and just saying that something is said somewhere in a paper
  isn't that helpful.
\item
  It could be useful for you if we think that the author did not in fact
  say the thing you are attributing to them. One thing that happens in
  courses like this is that students find something where an author says
  something somewhat different on page 12 to what they said on page 4.
  If we've primarily remembered the view on page 4, you might be
  entirely right in what you attribute to them, even if it differs from
  our memory.
\item
  Finally, a more 2024 reason. LLMs are really bad at this kind of
  granular citation. So this kind of detail is pretty convincing
  evidence that you're turning in your own work.
\end{enumerate}

\section{Generative AI Policy}\label{generative-ai-policy}

For this course, we don't have a blanket ban on using generative AI,
like UMichGPT. In theory, these tools could be useful editors, or
proof-readers, and I don't feel comfortable ruling out those usage. But
there are three important constraints on using generative AI if you do
use it. If I was in your position, complying with these constraints
would be more trouble than it's worth, and I would rather simply write
my own papers. But if you would use them, these are the constraints.

First, it must comply with the following constraints (which are based on
the ``Permitted Depending on Activity Type'' model on
https://genai.umich.edu/guidance/faculty/course-policies.)

The use of generative AI tools (e.g.~ChatGPT, Dall-e, etc.) is permitted
in this course for the following activities:

\begin{itemize}
\tightlist
\item
  Brainstorming and refining your ideas;
\item
  Checking grammar and style.
\end{itemize}

The use of generative AI tools is not permitted in this course for the
following activities:

\begin{itemize}
\tightlist
\item
  Impersonating you in classroom contexts.
\item
  Doing your part of collaborative work.
\item
  Writing a draft of a writing assignment.
\item
  Writing entire sentences, paragraphs or papers to complete class
  assignments.
\end{itemize}

Second, note that you've used any of these tools. Note which tool you
used (ChatGPT, UMGPT, editing software, etc.) and what you used it for.

Third, keep records of what prompts you entered. This is very important
if there is a suspicion that you've crossed the line from using it for
checking for grammar and style to getting it to write whole sentences.
(Automated checkers couldn't tell between these uses even if they
worked, which in any case they don't. But keeping a record of your
pre-AI text, and of your prompt, does show the difference.)

Fourth, include citations, even if they got wiped out by the generative
AI tools.

The basic principle I have here is that no one would get in trouble for
using the spelling or grammar checkers in word processing software, and
if you're using generative AI as updated spelling or grammar checkers, I
don't think you should get in trouble for this. (Though do note that not
everyone agrees - do not take this as a licence for using them in other
classes.) But there's no skill in just copying the essay prompt into a
chat window and copying the answer into a paper.

Finally, note that we have actually copied the essay questions into one
prominent chatbot, and the answers it gave were often completely wrong.
In one case it confused which authors had which view, in another it just
made up arguments that had nothing to do with the papers under
discussion. Maybe things will change in the near or far future, but for
now the tools give the same confident answer whether they have
accurately summarised a source document, or they are just making stuff
up. They are like that friend who knows a bunch of things, but thinks
they know a lot more than they actually do. And trusting such a friend
is risky.

\section{Disability}\label{disability}

The University of Michigan abides by the Americans with Disabilities Act
of 1990, Section 504 of the Rehabilitation Act of 1973, and other
applicable federal and state laws that prohibit discrimination on the
basis of disability, which mandate that reasonable accommodations be
provided for qualified students with disabilities.

If you have a disability, and may require some type of instructional
and/or examination accommodation, please contact me early in the
semester. If you have not already done so, you will also need to
register with the Office of Services for Students with Disabilities. The
office is located at G664 Haven Hall.

For more information on disability services at the University of
Michigan, go to \url{http://ssd.umich.edu}.

\section{Class Schedule}\label{class-schedule}

\subsection{Group Attitudes}\label{group-attitudes}

\subsubsection{Thursday, January 11}\label{thursday-january-11}

\begin{description}
\tightlist
\item[Topic]
Introduction
\item[Reading]
No New Reading
\end{description}

\subsubsection{Tuesday, January 16}\label{tuesday-january-16}

\begin{description}
\tightlist
\item[Topic]
Group Action
\item[Reading]
Margaret Gilbert, \href{https://philpapers.org/rec/GILWTA}{Walking
Together: A Paradigmatic Social Phenomenon}
\end{description}

\subsubsection{Thursday, January 18}\label{thursday-january-18}

\begin{description}
\tightlist
\item[Topic]
Group Action
\item[Reading]
Michael Bratman, \href{https://philpapers.org/rec/BRASCA}{Shared
Cooperative Activity}
\item[Recommended Reading]
Michael Bratman, \href{https://philpapers.org/rec/BRASI}{Shared
Intention}
\end{description}

\subsubsection{Tuesday, January 23}\label{tuesday-january-23}

\begin{description}
\tightlist
\item[Topic]
Group Belief
\item[Reading]
Jennifer Lackey, \href{https://philpapers.org/rec/LACWIJ}{What is
Justified Group Belief?} (The important sections to read are 1, 4-7, and
9. The paper is very long, and reading those sections should be plenty.)
\end{description}

\subsubsection{Thursday, January 25}\label{thursday-january-25}

No class; I'm away at a conference

\subsubsection{Tuesday, January 30}\label{tuesday-january-30}

\begin{description}
\tightlist
\item[Topic]
Group Justification
\item[Reading]
Jessica Brown, \href{https://philpapers.org/rec/BROGBA-3}{Group belief
and direction of fit}
\end{description}

\subsubsection{Thursday, February 01}\label{thursday-february-01}

\begin{description}
\tightlist
\item[Topic]
Merging Probabilities
\item[Reading]
Jeffrey Sanford Russell, John Hawthorne and Lara Buchak,
\href{https://philpapers.org/rec/RUSG}{Groupthink}
\end{description}

\subsection{Voting}\label{voting}

\subsubsection{Tuesday, February 06 and Thursday, February
08}\label{tuesday-february-06-and-thursday-february-08}

\begin{description}
\tightlist
\item[Topic]
Voting systems
\item[Reading]
Simon Hix, Ron Johnston, and Iain McLean
\href{https://www.thebritishacademy.ac.uk/publications/choosing-electoral-system/}{Choosing
an Electoral System}
\end{description}

\subsubsection{Tuesday, February 13 and Thursday, February
15}\label{tuesday-february-13-and-thursday-february-15}

\begin{description}
\tightlist
\item[Topic]
Arrow's Theorem
\item[Reading]
Michael Morreau,
\href{https://plato.stanford.edu/entries/arrows-theorem/}{Arrow's
Theorem}
\item[Recommended Reading]
John Geanakoplos, \href{https://www.jstor.org/stable/25055941}{Three
Brief Proofs of Arrow's Impossibility Theorem}
\end{description}

\subsubsection{Tuesday, February 20 and Thursday, February
22}\label{tuesday-february-20-and-thursday-february-22}

\begin{description}
\tightlist
\item[Topic]
Sen on Social Choice
\item[Reading]
Amartya Sen, \href{https://www.jstor.org/stable/1829633}{The
Impossibility of a Paretian Liberal}

Amartya Sen, \href{https://www.jstor.org/stable/117024}{The Possibility
of Social Choice}
\item[Recommended Reading]
Christian List,
\href{https://plato.stanford.edu/entries/social-choice/}{Social Choice
Theory}
\end{description}

\subsubsection{Tuesday, February 27 and Thursday, February
29}\label{tuesday-february-27-and-thursday-february-29}

Winter Break

\subsection{Games and Coordination}\label{games-and-coordination}

Note that in this unit, and this unit \textbf{only} it would be better
to do the reading \textbf{after} we discuss the material in class.

\subsubsection{Tuesday, March 05 and Thursday, March
07}\label{tuesday-march-05-and-thursday-march-07}

\begin{description}
\tightlist
\item[Topic]
Prisoner's Dilemma
\item[Reading]
Bonanno, §2.1 and 2.2

Robert Axelrod, \href{https://www.jstor.org/stable/173638}{More
Effective Choice in the Prisoner's Dilemma}
\item[Recommended]
Robert Axelrod, \href{https://www.jstor.org/stable/173932}{Effective
Choice in the Prisoner's Dilemma}

Robert Axelrod and William Hamilton,
\href{https://www.jstor.org/stable/1685895}{The Evolution of
Cooperation} (possibly the most cited humanities/social science article
ever written)

Robert Axelrod, \href{https://www.jstor.org/stable/1961366}{The
Emergence of Cooperation among Egoists}

Steven Kuhn,
\href{https://plato.stanford.edu/entries/prisoner-dilemma/}{Prisoner's
Dilemma}
\item[Interaction]
From this time on we'll be making heavy use of the experimental setups
on \href{https://veconlab.econ.virginia.edu/}{Veconlab}, and it is very
important that you have your account set up by then.
\end{description}

\subsubsection{Tuesday, March 12}\label{tuesday-march-12}

\begin{description}
\tightlist
\item[Topic]
Iterated Deletion
\item[Reading]
Bonanno, §2.5 and 2.6
\end{description}

\subsubsection{Thursday, March 14}\label{thursday-march-14}

\begin{description}
\tightlist
\item[Topic]
Backward Induction
\item[Reading]
Bonanno, §3.1 and 3.2
\end{description}

\subsubsection{Tuesday, March 19}\label{tuesday-march-19}

\begin{description}
\tightlist
\item[Topic]
Stag Hunt
\item[Reading]
Brian Skyrms, \href{https://www.jstor.org/stable/3218711}{Stag Hunt}
\end{description}

\subsubsection{Thursday, March 21}\label{thursday-march-21}

\begin{description}
\tightlist
\item[Topic]
Coordination and Risk
\item[Reading]
Kaushik Basu, \href{https://www.jstor.org/stable/2117865}{The Traveler's
Dilemma: Paradoxes of Rationality in Game Theory}
\end{description}

\subsubsection{Tuesday, March 26}\label{tuesday-march-26}

\begin{description}
\tightlist
\item[Topic]
Focal Points
\item[Reading]
Judith Mehta, Chris Starmerand Robert Sugden,
\href{https://www.jstor.org/stable/2118074}{The Nature of Salience: An
Experimental Investigation of Pure Coordination Games}
\end{description}

\subsubsection{Thursday, March 28}\label{thursday-march-28}

\begin{description}
\tightlist
\item[Topic]
Limits to Induction
\item[Reading]
Rosemarie Nagel, \href{https://www.jstor.org/stable/2950991}{Unraveling
in Guessing Games: An Experimental Study}
\end{description}

\subsection{Information Networks}\label{information-networks}

\subsubsection{Tuesday, April 02}\label{tuesday-april-02}

\begin{description}
\tightlist
\item[Topic]
Introducing Signaling Games
\item[Reading]
In-Koo Cho and David Kreps,
\href{https://www.jstor.org/stable/1885060}{Signaling Games and Stable
Equilibria}, §2
\end{description}

\subsubsection{Thursday, April 04}\label{thursday-april-04}

\begin{description}
\tightlist
\item[Topic]
The Market for Lemons
\item[Reading]
George Akerlof, \href{https://www.jstor.org/stable/1879431}{The Market
for ``Lemons'': Quality Uncertainty and the Market Mechanism}
\end{description}

\subsubsection{Tuesday, April 09}\label{tuesday-april-09}

\begin{description}
\tightlist
\item[Topic]
Job Market Signaling
\item[Reading]
Michael Spence, \href{https://www.jstor.org/stable/1882010}{Job Market
Signaling}, §§1-4
\end{description}

\subsubsection{Thursday, April 11}\label{thursday-april-11}

\begin{description}
\tightlist
\item[Topic]
Information Cascades
\item[Reading (read \textbf{after} class)]
Lisa Anderson and Charles Holt,
\href{https://www.jstor.org/stable/2951328}{Information Cascades in the
Laboratory}
\end{description}

\subsubsection{Tuesday, April 16}\label{tuesday-april-16}

\begin{description}
\tightlist
\item[Topic]
Epistemic Networks
\item[Reading]
Cailin O'Connor and Sanford Goldberg,
\href{https://plato.stanford.edu/entries/epistemology-social/\#NetwEpisMode}{Social
Epistemology}, §4.3
\end{description}

\subsubsection{Thursday, April 18}\label{thursday-april-18}

\begin{description}
\tightlist
\item[Topic]
Misleading with Truth
\item[Reading]
James Owen Weatherall, Cailin O'Connor, and Justin Bruner,
\href{https://www.journals.uchicago.edu/doi/abs/10.1093/bjps/axy062}{How
to Beat Science and Influence People: Policymakers and Propaganda in
Epistemic Networks}
\end{description}

\newpage

\subsection{Full List of Papers}\label{full-list-of-papers}

\begin{itemize}
\tightlist
\item
  George Akerlof, \href{https://www.jstor.org/stable/1879431}{The Market
  for ``Lemons'': Quality Uncertainty and the Market Mechanism}
\item
  Lisa Anderson and Charles Holt,
  \href{https://www.jstor.org/stable/2951328}{Information Cascades in
  the Laboratory}
\item
  Robert Axelrod, \href{https://www.jstor.org/stable/173638}{More
  Effective Choice in the Prisoner's Dilemma}
\item
  Robert Axelrod, \href{https://www.jstor.org/stable/173932}{Effective
  Choice in the Prisoner's Dilemma}
\item
  Robert Axelrod and William Hamilton,
  \href{https://www.jstor.org/stable/1685895}{The Evolution of
  Cooperation}
\item
  Robert Axelrod, \href{https://www.jstor.org/stable/1961366}{The
  Emergence of Cooperation among Egoists}
\item
  Kaushik Basu, \href{https://www.jstor.org/stable/2117865}{The
  Traveler's Dilemma: Paradoxes of Rationality in Game Theory}
\item
  Michael Bratman, \href{https://philpapers.org/rec/BRASI}{Shared
  Intention}
\item
  Michael Bratman, \href{https://philpapers.org/rec/BRASCA}{Shared
  Cooperative Activity}
\item
  Jessica Brown, \href{https://philpapers.org/rec/BROGBA-3}{Group belief
  and direction of fit}
\item
  John Geanakoplos, \href{https://www.jstor.org/stable/25055941}{Three
  Brief Proofs of Arrow's Impossibility Theorem}
\item
  Margaret Gilbert, \href{https://philpapers.org/rec/GILWTA}{Walking
  Together: A Paradigmatic Social Phenomenon}
\item
  In-Koo Cho and David Kreps,
  \href{https://www.jstor.org/stable/1885060}{Signaling Games and Stable
  Equilibria}
\item
  Simon Hix, Ron Johnston, and Iain McLean
  \href{https://www.thebritishacademy.ac.uk/publications/choosing-electoral-system/}{Choosing
  an Electoral System}
\item
  Steven Kuhn,
  \href{https://plato.stanford.edu/entries/prisoner-dilemma/}{Prisoner's
  Dilemma}
\item
  Jennifer Lackey, \href{https://philpapers.org/rec/LACWIJ}{What is
  Justified Group Belief?}
\item
  Christian List,
  \href{https://plato.stanford.edu/entries/social-choice/}{Social Choice
  Theory}
\item
  Judith Mehta, Chris Starmerand Robert Sugden,
  \href{https://www.jstor.org/stable/2118074}{The Nature of Salience: An
  Experimental Investigation of Pure Coordination Games}
\item
  Michael Morreau,
  \href{https://plato.stanford.edu/entries/arrows-theorem/}{Arrow's
  Theorem}
\item
  Rosemarie Nagel,
  \href{https://www.jstor.org/stable/2950991}{Unraveling in Guessing
  Games: An Experimental Study}
\item
  Cailin O'Connor and Sanford Goldberg,
  \href{https://plato.stanford.edu/entries/epistemology-social/\#NetwEpisMode}{Social
  Epistemology}
\item
  Jeffrey Sanford Russell, John Hawthorne and Lara Buchak,
  \href{https://philpapers.org/rec/RUSG}{Groupthink}
\item
  Amartya Sen, \href{https://www.jstor.org/stable/1829633}{The
  Impossibility of a Paretian Liberal}
\item
  Amartya Sen, \href{https://www.jstor.org/stable/117024}{The
  Possibility of Social Choice}
\item
  Brian Skyrms, \href{https://www.jstor.org/stable/3218711}{Stag Hunt}
\item
  Michael Spence, \href{https://www.jstor.org/stable/1882010}{Job Market
  Signaling}
\item
  James Owen Weatherall, Cailin O'Connor, and Justin Bruner,
  \href{https://www.journals.uchicago.edu/doi/abs/10.1093/bjps/axy062}{How
  to Beat Science and Influence People: Policymakers and Propaganda in
  Epistemic Networks}
\end{itemize}



\end{document}
